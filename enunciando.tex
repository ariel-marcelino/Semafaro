Desafio: Simulador de Radar de VelocidadeObjetivo:
Criar uma página web que simula um radar de velocidade com um semáforo. O objetivo é seguir as regras de trânsito: respeitar o semáforo, não ultrapassar a velocidade máxima permitida para cada via e não ficar parado no semáforo verde por muito tempo.
Requisitos:
Seleção de Via:
Um elemento <select> que permite ao usuário escolher entre diferentes vias com limites de velocidade específicos.
As opções devem ser:
Rua A - 30 km/h
Avenida B - 50 km/h
Rodovia C - 80 km/h
Semáforo:
O semáforo deve ter três luzes: vermelho, amarelo e verde.
O semáforo começa no vermelho.
O semáforo muda de vermelho para verde em 3 segundos.
O semáforo muda de verde para amarelo em um tempo aleatório entre 15 e 25 segundos.
O semáforo muda de amarelo para vermelho em 10 segundos.
Controle de Velocidade:
Dois botões: "Acelerar (X)" e "Desacelerar (C)".
Cada vez que o botão "Acelerar" é pressionado, a velocidade aumenta em 10 km/h.
Cada vez que o botão "Desacelerar" é pressionado, a velocidade diminui em 10 km/h.
A velocidade não pode ser negativa.
Regras do Semáforo:
Quando o semáforo está vermelho, o carro deve estar parado (0 km/h).
Se o semáforo está vermelho e o carro não está parado, o usuário recebe uma multa e o jogo reinicia.
Quando o semáforo está verde, o carro pode andar. Se a velocidade do carro ficar em 0 km/h por 10 segundos consecutivos com o semáforo verde, o usuário recebe uma multa e o jogo reinicia.
Quando o semáforo está amarelo, o usuário deve se preparar para parar.
Se o semáforo está vermelho e o carro está andando, o usuário recebe uma multa e o jogo reinicia.
Limite de Velocidade:
Se a velocidade do carro ultrapassar o limite permitido para a via selecionada, o usuário recebe uma multa e o jogo reinicia.
Passo a Passo para o Desenvolvimento1. Estrutura HTML
Criação do elemento <select>:
Adicionar opções de vias com limites de velocidade.
Criação do semáforo:
Usar div para representar cada luz do semáforo.
Criação dos botões de controle:
Botões para acelerar e desacelerar.
Criação da área de status:
Mostrar a velocidade atual e o estado do carro (parado ou andando).
2. Estilo CSS
Estilo do semáforo:
Estilizar as luzes para que sejam claramente visíveis.
Estilo dos botões e status:
Estilizar os botões e a área de status para que sejam fáceis de usar e ler.
3. Lógica JavaScript
Inicialização das variáveis:
Variáveis para controlar a velocidade, o limite de velocidade, o estado do semáforo e os temporizadores.
Funções de controle do semáforo:
Funções para mudar a cor do semáforo e controlar o tempo de cada estado.
Funções de controle da velocidade:
Funções para aumentar e diminuir a velocidade do carro.
Funções de verificação de regras:
Funções para verificar se o carro está respeitando as regras de velocidade e semáforo.
Função de reinicialização do jogo:
Função para reiniciar o jogo em caso de infração.